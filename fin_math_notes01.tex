\documentclass{article}
%为注释符号,下一行为导包以支持中文显示
\usepackage[UTF8]{ctex}
%支持绘图和图片相关指令...
\usepackage{graphicx}
% 这里是导言区
\title{金融中的数学\\How We Understand Finance in Math}
\author{Yueyue Liu}
\date{\today}

\begin{document}
\maketitle
\section{基础金融知识与数学模型}
\subsection[UTF8]{商业银行的作用}
\subsubsection[UTF8]{Commercial bank has the ability to "produce" money.\\(商业银行的“造钱”能力)}
商业银行能够“造钱”的一个条件是:有别于其他类型的金融机构,商业银行是经过央行,即中国人民银行批准的,可以吸收公众存款的金融机构。\newline
\textbf{存款准备金}指商业银行的初级存款中不能用于放贷的部分。如果当下的存款准备金率为20\%,那一个储户将10000元存入工商银行后,工商银行必须将其2000元存入中国人民银行,以备该储户提款的需要。此时,工商银行手中还剩8000元可以用于放贷。假设此时一个人向银行借了这8000元并且到一家店里买了一台iPad,这家店随后将这8000元存入建设银行。这个时候,建设银行吸收了8000元的存款,按20\%的存款准备金率它需要将其中的8000*0.2=1600元存入中国人民银行,并可将剩余的6400元贷出。倘若借走这6400元的人买了一部iPhone手机,而后卖方将这6400元存入交通银行,则交通银行须将其中的20\%也就是1280元存入央行并可将5120元贷出...像这样循环往复,银行不停的放出贷款,而又有客户将收入存入银行。那么最终(即商业银行无钱可贷时),涉及其中的所有商业银行将各自增加存款10000元,8000元,6400元,5120元...,各自增加(扣除准备金后)贷款8000元,6400元,5120元,4096元...。理论上,银行系统账面上的新增存款总额是50000元,新增贷款总额是40000元。也就是说,货币总量增加到50000元而法定准备金为50000-40000=10000元。此时我们发现准备金刚好为预设值20\%。


\end{document}